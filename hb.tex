%
% Version 1: 9/24/13
%
\documentclass{article}
\usepackage[UTF8]{ctex}

\usepackage{amsmath}
\usepackage{amsfonts}
\usepackage{amsthm}
\usepackage[colorinlistoftodos]{todonotes}
\usepackage{pifont}
\usepackage{mdframed,color}
\usepackage{draftwatermark}
\SetWatermarkText{mariko-33} % the Text
\SetWatermarkLightness{0.9} % the lightness from 0 to 1, default 0.8
\SetWatermarkScale{1.0} % the scale, default 1.2


\newtheoremstyle{statement}{3pt}{3pt}{}{}{\bfseries}{:}{.5em}{}

\theoremstyle{statement}
%\newtheorem*{atmProp}{Proposition}

%\newenvironment{atmProof}{\noindent\ignorespaces\paragraph{Proof:}}{\hfill %\ding{122}\par\noindent}

%%%%% You should not need to edit anything above this line. %%%%%

\title{HAHN-BANACH THEOREM} % Put the title of the assignment here...
%%%\author{Your Name}     % ... and your name here.
\newtheorem{theorem}{Theorem}
%\newenvironment{proof}{\begin{proof}}{\end{proof}\qed}

\begin{document}

	\maketitle
	
	%%%%% You can copy the following segment of code for each proof in the assignment. %%%%%
	
	%\begin{mdframed}[
	%%	innerleftmargin=10pt,
	%	innertopmargin=10pt,
	%	innerbottommargin=10pt,
%	skipabove=15pt,
		%skipbelow=15pt
	%	]
	%	\textbf{6b) Let $a$ and $b$ be rational numbers such that %$\sqrt{ab}$ is irrational. Prove that $\sqrt{a}+\sqrt{b}$ is %irrational.}
	%\end{mdframed}
	
	\begin{theorem}[HB theorem]
		设$X$是实数域上的线性空间,$p$是定义在$X$上的实值函数,并且设$p$是次线性函数,具有以下性质:
		\\
		
		\begin{enumerate}
		\item[\emph{(i)}] \emph{正齐次性(Positive  homogeneity):}
	
		
		$$p(ax)=ap(x) \qquad  \forall  a > 0.  \eqno{(1)}$$

		
		\item[\emph{(ii)}] \emph{次可加性(Subadditivity):}
		

		$$p(x+y) \leq p(x)+p(y) \qquad \forall  x,y\in X. \eqno{(2)}$$ 

			
			
		\end{enumerate}
	
		设$Y$是$X$的子空间,$l$是定义在$Y$上的线性函数,并且被$p$约束:
		$$l(y) \leq p(y) \qquad \forall y \in Y. \eqno{(3)}$$ 
		
		
		那么$l$可以作为线性函数延展到整个$X$,且被$p$所约束:
		$$l(x) \leq p(x) \qquad \forall x \in X. \eqno{(3')}$$

		
		
	\end{theorem}
	
	\begin{proof}[proof]
		设集合
		\begin{center}
		$\mathcal{S}=\{(Z,f),$其中$Z$是$X$的子空间且有$Y\subset Z,f:Z\to \mathbb{R}$是线性函数$, f(x)  \leq p(x),\forall  x \in  Z, f(y)=l(y), \forall  y \in  Y \}.$
		\end{center}
		首先证明$\mathcal{S}$的存在性,即线性函数$f$的存在性.
		\\
		
		$f$存在性证明:为了证明$f$的存在性,我们只需要将线性函数$l$延展到$X$的子空间$Z$即可,即:
		$$l(y+az)=l(y)+al(z) \leq p(y+az)  \qquad \forall  y\in Y , \,\, \forall  a \in \mathbb{R} \eqno{(4)}$$
		其中$Z$是$X$的子空间,且$Z=<Y,z>$,即$Z$是由$Y$和$z$生成,具有以下形式:
		
		$$y+az, \qquad \forall  y \in Y , z \in X \,\, and \,\, z \notin Y , a \in \mathbb{R}$$
		则当$a=0$时,$(4)$显然成立,现在讨论$a\neq 0$时的情况,由于$p$存在正齐次性,所以只需要验证$a=\pm 1$即可,即:
		$$l(y)+l(z) \leq p(y+z), \quad l(y')-l(z) \leq p(y'-z) \quad \forall  y,y' \in Y , \forall  z \in Z \eqno{(5)}$$
		假设$l(z)$存在且$(5)$成立,那么有:
		$$p(y+z)-l(y) \geq l(z) \geq l(y')-p(y'-z) \eqno{(6)}$$
		即:
		$$p(y+z)+p(y'-z) \geq l(y)+l(y') \eqno{(6')}$$
		而$y+y' \in Y$,根据$(3)$有:
		$$p(y+y') \geq l(y+y') \eqno{(7)}$$
		又由次可加性可知
		$$p(y+y')=p(y+z+y'-z) \leq p(y+z)+p(y'-z) \eqno{(8)}$$
		结合$(7)(8)$可知$(6')$成立,所以可知$l(z)$存在,故$f$的存在性可证明.
		\\
		
		于是集合$\mathcal{S}$存在.对于任意的$(Z_i,f_i),(Z_j,f_j) \in \mathcal{S}$,不妨令$Z_i \subseteq Z_j$,则设:
		$$(Z_i,f_i) \preceq (Z_j,f_j) \iff f_{j}|_{Z_i}=f_i, \, Z_i \subseteq Z_j$$
		成立偏序关系.
		
		设对于任意的$(Z_i,f_i) ,(Z_j,f_j)\in \mathcal{S}$,令:
		$$(Z_i,f_i) \preceq (Z_j,F_j), \forall i\leq j \eqno{(9)}$$
		同时令$Z=\cup{Z_k}, \forall (Z_k,f_k) \in \mathcal{S}$,显于$(Z,f)$中的$f$满足$(3)$,并且对于所有的$k$满足:
		$$(Z_k,f_k) \preceq (Z,f) \eqno{(10)}$$
		由$(9)(10)$可知满足Zorn引理,所以必然存在一个最大的延展,由之前的讨论可知,这个最大的延展必然是整个线性空间$X$.
		
		
	\end{proof}
	
	%%%%%%%%%%
	
	
	
\end{document}